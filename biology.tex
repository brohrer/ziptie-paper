\section{Biological Motivation}
\label{sec:bio}

Ziptie grew out of a computational neuroscience thought experiment.
Imagine that 

\begin{itemize}
\item{
The firing rate of a neuron can be represented as a value between
zero (nearly no activity) and one (a maximum, tonic rate).
This can be represented as a fuzzy categorical variable.
If a neuron is connected to, say, a pain receptor, then a zero
would represent no pain signal, and a one, the highest amount
of pain the receptor can sense.
}
\item{
Sensory neurons carry no explicit
information about what physical phenomena
they are sensing.
}
\item{
The brain is a collection point for sensory neurons of all types,
where it organizes them into useful features. For instance, optical
neurons are organized into feature detectors for edges.
}
\item{
The brain does this using only the neurons' activation history.
}
\item{
Since it is agnostic to sensory modality, the brain naively applies
the same methodology (algorithm) to all incoming sensory neurons.
}
\item{
The features created are sparse, that is, only a small number of
neurons can make a contribution to them. 
}
\item{
The set of neurons that can contribute to a feature
is created according to a variation of
\href{https://en.wikipedia.org/wiki/Hebbian_theory}{Hebbian theory}
in which neurons that fire at the same time become more closely 
associated over time.
}
\item{
An active feature in turn inhibits the neurons that contribute to it,
so that they can't simultaneously contribute to other features.
The collection of neurons' activities is preferentially represented
in higher level features.
}
\end{itemize}

This was the line of wondering, coupled with a lot of trial and error
in simulations, that led to the Ziptie algorithm.
