\section{Bookkeeping}
\label{sec:bookkeeping}

\subsection{Versions}
\label{subsec:versions}

The \textbf{\href{
https://codeberg.org/brohrer/ziptie-paper/src/branch/main/ziptie.pdf}
{latest version}}
of this document and all the files needed to
render it are in \textbf{\href{
https://codeberg.org/brohrer/ziptie-paper}
{this Codeberg repository}}. There's a backup copy in \textbf{
\href{https://github.com/brohrer/ziptie-paper}
{this repository on GitHub}}.
 \textbf{
\href{https://gitlab.com/brohrer/ziptie-paper}
{this repository on GitLab}}.

I don't expect this doc to ever be done. I'm always learning new things,
or thinking of a better way to explain something, or I do a new
piece of work I can't help myself from including. And there's always
one more bug.
Since it's a git repository, you are free to browse past commits to watch
the evolution, but I'll try to keep a running record of important updates
here.

\begin{itemize}
\item{\textbf{\href{
https://codeberg.org/brohrer/ziptie-paper/src/tag/2023-12-02/ziptie.pdf}
{December 2, 2023}}. Rough outline of how Ziptie works
and how it's related to the rest of the algorithmic world.}
\end{itemize}

\subsection{History}
\label{subsec:history}

Ziptie didn't start life in its current form. It actually has a long
and very boring history.

\begin{itemize}

\item{\textbf{2011}. As the method was taking shape, I published a 
flurry of posters and write-ups in small conferences: 
\href{https://github.com/brohrer/publications/blob/main/Rohrer11BeccaFunctionalModel.pdf}
{GCNC},
\footnote{
Rohrer, B., Morrow, J.D., Rothganger, F., Xavier, P. (2011)
\textit{BECCA: A functional model of the human brain
for arbitrary task learning}.
Grand Challenges in Neural Computation 2011.
}
\href
{https://github.com/brohrer/publications/blob/main/Rohrer11ImplementedArchitectureFeature.pdf}
{AGI},
\footnote{
Rohrer, B. (2011) \textit{An implemented architecture for
feature creation and general reinforcement learning}.
Workshop on Self-Programming in AGI Systems, AGI 2011.
}
\href
{https://github.com/brohrer/publications/blob/main/Rohrer11DevelopmentalAgentLearning.pdf}
{ICDL/EpiRob},
\footnote{
Rohrer, B. (2011) \textit{A developmental agent for learning features,
environment models, and general robotics tasks}. ICDL/Epirob 2011.
}
\href
{https://github.com/brohrer/publications/blob/main/Rohrer11BiologicallyInspiredFeature.pdf}
{BICA},
\footnote{
Rohrer, B. (2011) \textit{Biologically inspired feature creation
for multi-sensory perception}. BICA 2011.}
\href{https://github.com/brohrer/publications/blob/main/Rohrer12BeccaReintegratingAi.pdf}
{AAAI Symposium on Designing Intelligent Robots}.
\footnote{
Rohrer, B. (2012) \textit{BECCA: Reintegrating AI for
natural world interaction}. AAAI Spring Symposium on Designing
Intelligent Robots: Reintegrating AI 2012.
}
Originally Ziptie was developed as part of a larger project,
a cognitive architecture originally called the Brain Emulating
Cognition and Control Architecture. It shows up in that context
until it gets split out on its own later. The cognitive architecture
undergoes a lot of evolution, and other components come and go, but
Ziptie is the closest thing it has ot a fixed point.
}

\item{\textbf{2012-01-14}.

\href{
https://github.com/brohrer/openBECCA/blob/3a9b3902eeb78c61585efa6860ef9862298f562f/labBECCA/grouper_initialize.m}
{The oldest version}
of the code I can find.
At this point Ziptie was called Grouper
and was written in MATLAB. A lot of details have changed since this point,
but the acculumulation of coactivation as a clustering mechanism has not.
\href{https://www.linkedin.com/in/mattchapmansoftwareengineer/}{Matt Chapman}
was an early collaborator and helped me transition the code from the private
research repo I'd been developing in to something more public.
}

\item{\textbf{2012-02-20}.
This is
\href{https://github.com/alito/becca/commit/6eea91447608a1a6280ce915a26a18a85a57ef1c#diff-2106149c513c2a9fb07b17281f3830e992856b0c553b348f398b5d0fc3f96bc1}
{the first incarnation} of the code in Python. It was written
by
\href{https://www.linkedin.com/in/alejandro-dubrovsky-90b9b833/}{Alejandro Dubrovsky}
(GitHub user name \textit{alito}) who generously ported the MATLAB code
to Python as part of an epic weekend grind fest.
}

\item{\textbf{2012-04-20}.
I started using
\href{https://github.com/alito/becca/commit/dae6fa355c6bdcf9b5e2863d3b93dafa4fc4d5be#diff-2106149c513c2a9fb07b17281f3830e992856b0c553b348f398b5d0fc3f96bc1}
{the term ``coactivation"}
in the code and documentation.
}
\item{\textbf{2012-06-26}.
``Grouper" is renamed ``
\href{https://github.com/alito/becca/commit/6e7f968f283e287917cbbc9520297b833ad5d3aa#diff-e521b0d7fc82aa487ff7542d8a2672c0074ee1ae38705a48676fbad625e39bed}
{Perceiver}".
}

\item{\textbf{2012-06-26}.
``Perceiver" is renamed ``
\href{https://github.com/alito/becca/commit/dbb0374b164b6a5725b5bdf7d9eb208ddd82f855#diff-9338f7a12f369b96ae4414274afd08ce028c84758d252e683d937e6a56c48c9a}
{Map}".
}

\item{\textbf{2013-05-09}.
``Map" is renamed ``
\href{https://github.com/alito/becca/blob/ba07dde43f5a24fb35c1eabcc756a2cdc799287d/src/scripts/agent/ziptie.py}
{Ziptie}".
}

\item{\textbf{2015-01-13}.
The home repository for the code is moved to
\href{https://github.com/brohrer/robot-brain-project/blob/bc41b26373290acb6be0c9069be6ac6f3485b106/core/ziptie.py}
{brohrer/robot-brain-project}.
}

\item{\textbf{2015-06-10}.
\href{https://github.com/brohrer/robot-brain-project/commit/3815c5a10cf256e9f61eeec1a34f3b60f55d107f#diff-f47908520ba430339ff71c0fb1052bc986dafdd875ad83c9083ed4687f469945}
{I started using Numba} to get everything to run faster.
}

\item{\textbf{2018-10-01}.
This is
\href{https://github.com/brohrer/robot-brain-project/blob/53e2b52ab10e4ed7f92b3d488c49a6138d29a878/becca/ziptie.py}
{the last commit}
in the GitHub robot-brain-project repository.
}

\item{\textbf{2018-11-08}.
The
\href{https://github.com/brohrer/ziptie}
{Ziptie code} is split out into its own repository.
}

\end{itemize}

\subsection{Citations}
\label{subsec:citations}

If you end up using Ziptie in your work, give it a shout out.
Here's an APA example you can copy and paste. (You may have to fiddle with
the dates.)

Rohrer, B. (2023). Ziptie: Learning Useful Features [White Paper].
Retrieved \today, from
\href{https://brandonrohrer.com/ziptie}{https://brandonrohrer.com/ziptie}

If you're so inclined, drop me a note too at brohrer@gmail.com.
I love to hear about how
Ziptie is being used. It gives me ideas for how to make it better.

\subsection{Licensing}
\label{subsec:license}

The text, figures, equations, and methods described in this paper
are published under the CC0 "No Rights Reserved" license.
From the
\href{https://creativecommons.org/public-domain/cc0/}{Creative Commons} description,
CC0 "enables scientists, educators,
artists and other creators and owners of copyright- or database-protected
content to waive those interests in their works and thereby place them as
completely as possible in the public domain, so that others may freely
build upon, enhance and reuse the works for any purposes without restriction
under copyright or database law."

\begin{figure}[ht]
\vskip 0.2in
\begin{center}
\centerline{\includegraphics[width=1.0in]{images/cc-zero.png}}
\label{fig:cc0}
\end{center}
\vskip -0.2in
\end{figure}

